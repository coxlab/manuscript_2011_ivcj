Face recognition has long been, and continues to be, a highly active area of research \cite{belhumeur2002eigenfaces,yang2002kernel,vasilescu2002multilinear,zhao2003face,he2005face,hua2007face,hua2009robust,guillaumin2009you,hua2009robust,wright2009implicit,zou2007comparative}.
In recent years, interest in the problem of \emph{unconstrained} face
recognition has grown in the community, driven in large part by the creation of
the \emph{Labeled Faces in the Wild (LFW)} \cite{huang:lfw} test set, which has
provided a standardized benchmark against which to measure progress.  While
face recognition research \foreign{per se} has a long and rich history, much
work prior to the last decade was focused on face recognition in relatively
constrained environments (e.g. posed photographs, under controlled  lighting
conditions \cite{orl,yale,cvl,ar,phillips2000feret,gross2009multi}).  More
recently, thanks in large part to the rise of the internet, it has become
possible to assemble large collections of face images ``in the wild'' in the
sense that they come from a wide variety of sources and were not posed for the
purpose of research.  While this set has proven to be quite challenging, large
strides have been in made in recent years towards higher performance
\cite{pinto:eccv08,pinto:cvpr09,taigman:bmvc09,wolf:accv09,kumar:iccv09,cao2010face}.

While a variety of different approaches to the \emph{LFW} set have been taken,
a common feature of most approaches is the use of some low-level visual feature
set, such as SIFT \cite{sift,luo2007person}; HOG
\cite{dalal2005hog,albiol2008face}; or LBP
\cite{ahonen2004face,ahonen2006face} that transforms raw pixels values into a
better form for subsequent processing.  While individual algorithms often do
not depend critically on the choice of a particular feature representation
used, the choice of features used does frequently play a key role in
determining performance.  Meanwhile, there are only a handful of visual feature
representations in common use, and arguably less attention has been paid to
developing new or better features.

One potentially promising source for new, more complex visual feature representations is the
class of ``biologically-inspired'' representations.
Biologically-inspired approaches seek to build artificial visual 
systems that capture aspects of the computational architecture of the brain, in the hope of
eventually mimicking its computational abilities. Such efforts to model visual
computations done by the brain have a long history, at least dating back to
Fukushima's Neocognitron (1980; \cite{fukushima1980neocognitron}).  More recent
experiments with biologically-inspired models have shown them to be highly
competitive in a variety of different face and object recognition
contexts \cite{serre2007ror,mutch2008ocr,pinto:plos08,pinto:eccv08,jarrett-iccv-09}.

However, the range of possible feature representations that would count as ``biologically-inspired'' 
is broad, and it is not clear which particular instantiations of biologically-inspired
ideas are best for a given task.  Pinto et al. \cite{pinto:plos09} previously demonstrated a 
high-throughput screening approach for biologically-inspired algorithms, wherein a large
number of possible candidate models  from an inclusive model family are considered, 
and the best performing models are ``skimmed off the top'' 
and evaluated further.  However, while that work showed success with synthetic test 
images, it has not been known to date whether models from this class are competitive with 
current state-of-the-art approaches on standard face and object recognition test sets.

Here we present a modified large-scale feature search procedure that simplifies and accelerates the
search procedure described in \cite{pinto:plos09}, with the goal of generating feature
representations tailored for unconstrained face recognition, as embodied by the
\emph{LFW} test set.  Multiple complimentary representations are further derived through
training set augmentation, alternative face comparison functions, and feature set searches 
with a varying number of model layers.  These individual feature representations are then 
combined using kernel techniques to achieve even better performance.
We show that our approach yields multiple feature sets that 
outperform previous state-of-the-art approaches on the \emph{LFW} set,
even while requiring less training data and using simpler machine learning backends.
Finally, as a complement to experiments with natural face images, we characterize the tolerance of the resulting representations to various kinds of image transformations using rendered face images undergoing known view transformations.
In addition to providing evidence for the utility of large-scale feature search for standard 
``real world'' test sets, these results emphasize the value of good underlying 
representations and point a path forward in the generation of new, more powerful visual features.

