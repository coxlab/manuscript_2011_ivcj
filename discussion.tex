
Our results provide more evidence that biologically-inspired models represent a promising
and powerful direction in face recognition research. Individual models from this
class are able to achieve good performance (e.g. around 77\% for \emph{V1-like} models,
84\% for \emph{HT-L3}), and blends of these models achieve more than 88\%
correct performance, placing it amongst the best performing systems on this challenge set.

Consistent with expectations, progressively more complex, multi-layer models are
able to outperform the simpler \emph{V1-like} model. Whether this higher
performance is due to a greater ability to tolerate image variation (one of the
original purposes for the construction of the \emph{HT-L3} model
class\cite{pinto:plos09}) or some other factor remains to be seen. It should be
noted that the \emph{HT-L2} and \emph{HT-L3} models used here were substantially
simplified from those present in \cite{pinto:plos09}, in that they did not have
structured filter kernels, nor were they subjected to any unsupervised learning.
Whether adding these features back will result in higher levels of performance
is an important future research question.

While there still remains substantial room for improvement, concerns that
the \emph{LFW} set does not necessarily accurately reflect the ``full'' problem
of unconstrained face recognition
remain \cite{pinto:eccv08,pinto:cvpr09,kumar:iccv09}.  \emph{LFW} includes only
a handful of examples per individual, and these photographs were often taken in
the same setting and at the same event.  Furthermore, Kumar et
al. \cite{kumar:iccv09} showed that human observers were able to perform at
greater-than-90\% correct even when the faces themselves were masked out of the
test images, indicating that the backgrounds in the \emph{LFW} are more than
sufficient for solving the task at a level higher than the current machine state
of the art.

An analysis of the errors made by our models provides some clues about which
parts of the \emph{LFW} set are difficult and which ones are not.  Our models
failed on remarkably similar sets of face pairs, indicating that a common core
of ``hard'' images may exist within the larger \emph{LFW} set.  A striking,
albeit anecdotal, observation is that common error cases are dominated by misses
when the same individual is shown in differing views and by false positives
when two different individuals are compared while viewed from a similar angle
(e.g. Fig. \ref{fig:error_examples}).  An important feature of the \emph{LFW}
set is that faces must be detected by a Viola-Jones face detector in order to be
included in the set, and this effectively restricts the range of face views that
enter into the set (i.e. there is a bias towards frontal views).  We hypothesize
that those more off-axis views that do manage to pass the face detection filter
will present a particularly difficult challenge for a system trained on
the \emph{LFW} set.  The low-level (e.g. pixel-level) difference between two
different views of the same individual can easily be larger than the low-level
differences between two individuals in a similar pose.  A system that is not
specially designed to tolerate this kind of variation will have a high false
alarm rate on trials where two different individuals are seen in the same pose
and a high miss rate where the same individual is compared across different
poses.  At the same time, if the \emph{LFW} set contains a relatively small
fraction of these off-axis faces, then a system trained exclusively on
the \emph{LFW} set will face difficulty learning to tolerate these cases, even
if that system has the capability to learn such tolerance in principle.

At the same time, experiments with synthetic faces suggest that the underlying 
representations (particularly of the best three layer model) can indeed tolerate
significant variation in face pose, provided that adequately diverse training data
is available.  More generally, the steady progression towards larger 
and larger tolerance to variation when more processing layers are added to the 
model provides encouragement that hierarchical models such as these provide 
a promising path forward in the construction of transformation-invariant 
representations.

As continued research manages to chip away at the remaining ``performance gap''
between human and machines on the \emph{LFW} set, increased attention will need
to be paid to whether \emph{LFW} truly represents the problem of interest.  On
one hand, as long as some performance gap exists, the set is obviously valid at
a basic level.  However, the question remains whether a ``fuller'' formulation
of the problem (i.e. more natural, less filtered) might lead to faster progress.

